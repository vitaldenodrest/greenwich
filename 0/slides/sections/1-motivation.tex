\subsection{The Helmholtz equation}

\begin{frame}
  \frametitle{The Helmholtz equation}

  Our studies will tackle the Helmholtz equation:

  \begin{block}{The homogeneous Helmholtz equation}
    \vspace{-0.6cm}
    \begin{align*}
      \Delta u + k^2 u = 0
    \end{align*}
    \vspace{-0.6cm}
  \end{block}

  It is possible to account for sources using $f$,
  a function with compact support:

  \begin{alertblock}{The inhomogeneous Helmholtz equation}
    \vspace{-0.6cm}
    \begin{align*}
      \Delta u + k^2 u = f
    \end{align*}
    \vspace{-0.6cm}
  \end{alertblock}

\end{frame}


\subsection{Wave propagation}

\begin{frame}
  \frametitle{The wave equation}

  The main interest for the Helmholtz equation arises from the study of wave propagation.\\
  For media that are homogeneous, isotropic, linear et non dispersive:

  \begin{block}{The wave equation}
    \vspace{-0.6cm}
    \begin{align*}
      \pdv[2]{p}{t} &= c^2 \Delta p
    \end{align*}
    \vspace{-0.5cm}
  \end{block}

  $\Delta = \nabla^2$ is the Laplacian, a differential operator. \\
  $p$ is a physical phenomenon propagated through one of the aforementioned media at a speed $c$. \\
  It depends on both space $\mathbf{r}$ and time $t$.


\end{frame}


\begin{frame}
  \frametitle{Variable seperation}

  Assuming the solution is separable in time and space ($ p(\mathbf{r}, t) = u(\mathbf{r})T(t) $),
  the wave equation can be rewritten as such:

  \[
    \frac{\Delta u}{u} = \frac{1}{c^2} \odv[2]{T}{t}
  \]

  The left-hand side only depends on space and the right-hand side only depends on time.
  In order to be equal in any situation, both members need to be equal to the same constant.
  This constant is set to $ - k^2 $ for convenience:
  \begin{align*}
    \frac{\Delta u}{u} &= - k^2 \\
    \frac{1}{c^2 T} \odv[2]{T}{t} &= - k^2
  \end{align*}

  The first equation, relative to space, is the homogeneous medium Helmholtz equation.

\end{frame}


\subsection{Fourier transform}


\subsection{Exotic equations}