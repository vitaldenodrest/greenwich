\documentclass{article}

\usepackage{amsmath}
\usepackage{amssymb}
\usepackage[
    colorlinks=true,
    linkcolor=blue,
    citecolor=red,
    urlcolor=cyan,
    filecolor=magenta,
]{hyperref}
\usepackage{cleveref}

\crefname{equation}{Eq.}{Eqs.}
\Crefname{equation}{Eq.}{Eqs.}
%\creflabelformat{equation}{#2#1#3}

\begin{document}

\section{Problem statement}

\subsection{Strong formulation}
We will work on a two-dimensional Helmholtz problem.
Let $\Omega \subset \mathbb{R}^2$ be an open bounded region whose border $\delta \Omega$ is sufficiently smooth.
Our strong formulation is the following:
$$
\text{Find } u \text{ such that: }
\begin{cases}
\Delta u + k^2 u = -f & \text{in } \Omega \\
u = 0 & \text{on } \partial \Omega
\end{cases}
$$
$\Delta = \nabla^2$ is the Laplace operator, $u$ could be the unknown acoustic pressure field, $k$ is the wavenumber and $f$ is a source term.
In general, $u$ and $k$ can be complex-valued. However, the paramter studies will be conducted on real-valued cases, with $k>0$.
The extension to complex-valued cases should be straightforward.

\subsection{Galerkin formulation}
The usual Galerkin formulation for our problem is the following:
$$
\text{Find } u^h \in \mathcal{V}^h 
\text{ such that }
\forall v^h \in \mathcal{V}^h , ~
a_G(u^h, v^h) = \left\langle f, v^h  \right\rangle_{L^2(\Omega)}
$$
Where $\mathcal{V}^h \subset H_0^1(\Omega)$ is finite-dimensional,
$\left\langle u, v  \right\rangle_{L^2(\Omega)}$ is the $L^2(\Omega)$ scalar product,
and $a_G$ is the following sesquilinear form:
\begin{align*}
a_G: H_0^1(\Omega)^2 &\to \mathbb{C} \\
(u, v) &\mapsto \left\langle \nabla u, \nabla v  \right\rangle_{L^2(\Omega)} - k^2 \left\langle u, v \right\rangle_{L^2(\Omega)}
\end{align*}
We will use a mesh method to build $\mathcal{V}^h$, partitioning $\Omega$ in mutually exclusive elements.

\subsection{GLS}
The Galerkin/least-squares formulation uses the previous formulation and adds other terms whose purpose is to minimize the square of the residue over the element interiors $\tilde{\Omega}$.
Their contribution is weighted by a parameter $\tau$ :
$$
\text{Find } u^h \in \mathcal{V}^h 
\text{, }
\forall v^h \in \mathcal{V}^h , ~
a_{GLS}(u^h, v^h) = \left\langle f, v^h \right\rangle_{L^2(\Omega)} + \left\langle \tau f, \mathcal{L}v^h \right\rangle_{L^2(\tilde{\Omega})}
$$
Where $\mathcal{L.} = \Delta. + k^2.$ is the Helmholtz operator, and $a_{GLS}$ is the follwing sesquilinear form:
\begin{align*}
a_{GLS}: H_0^1(\Omega)^2 &\to \mathbb{C} \\
(u, v) &\mapsto a_G(u,v) + \left\langle \tau \mathcal{L}u, \mathcal{L}v \right\rangle_{L^2(\tilde{\Omega})}
\end{align*}

\section{Dispersion analysis and optimal parameter}
The Helmholtz equation suffers from the pollution effect,
causing numerical wavenumbers to lose accuracy with the physical wavenumber increasing.

\paragraph{Exact solutions}
When considering the sourceless ($f=0$) and free-space version of our problem,
the exact solutions are $\forall x \in \mathbb{R}, ~ u(x) = C e^{i(k_x x + k_y y)}$.
With $k^2 = k_x^2 + k_y^2$
Our studies will revolve around this case.

\subsection{Dispersion analysis}

\subsubsection{Numerical resolution}
\label{GalerkineNumRes}
Using bilinear interpolation with nodal shape functions $N_{(i,j)}$
that respect the partition of unity and Kronecker delta properties
yields the linear system
$AU^h=0$,
where $U^h$ is a the vector of nodal values and $A$ is the impedence matrix. \\
When using our Galerkine formulation, the matrix is
$$
A_{G,(i,j)(l,m)} = \sum_{e \in E} K_{(i,j)(l,m)}^e - k^2 M_{(i,j)(l,m)}^e
$$
where $E$ is the set of elements whose boundary contain both nodes of nodal coordinates $(i,j)$ and $(l,m)$,
and $K_{(i,j)(l,m)}^e$ and $M_{(i,j)(l,m)}^e$
are respectively stiffness and masse terms assembled using shape functions $(i,j)$ and $(l,m)$ over the element $e$.

\subsubsection{Dispersion relation}
\label{GalerkineDispRel}
We study, without loss of generality,
the stencil formed by an interior node $(0,0)$ and its eight neighbours.
The four distances from the center node are denoted $h_{\xi}^{\pm}$, with $\xi$ the dimension ($x$ or $y$) and $\pm$ the direction.
The elements are rectangular. \\
For convenience, we might use
$h_{\xi}^{\pm} = \alpha_{\xi}^{\pm} h_{\xi}^{\pm}$ and $2h_{\xi} = h_{\xi}^- + h_{\xi}^+$. \\
The "dispersion relation" is the line $0$ of the system:
\begin{equation*}
\sum_{i,j\in \{-1, 0, +1\}}A_{(0,0)(i,j)}U_{(i,j)}^h\label{eq:stenicl}
\end{equation*}
In the following equations for convenience, $\{-, +\}$ could also be used to index neighboring nodes using directions. \\
The values of the four corner coefficients are, with $i, j \in \{-, +\}$:
\begin{equation}
    A_{(0,0)(i,j)} = \frac{k^2 h_{x}^i h_{y}^j}{36} + \frac{h_{x}^i}{6 h_{y}^j} + \frac{h_{y}^j}{6 h_{x}^i}
\end{equation}
The values of the four side nodes coefficients are:
\begin{align}
    &A_{(0,0)(\pm, 0)} = \frac{k^{2} h_{x}^{\pm} h_{y} }{9}
    + \frac{2 h_{y}}{3 h_{x}^{\pm}}
    - \frac{h_x^{\pm}}{3 \alpha_{y}^{-} \alpha_{y}^{+} h_{y}} \\
    &A_{(0,0)(0,\pm)} = \frac{k^{2} h_{x} h_{y}^{\pm}}{9}
    + \frac{2 h_{x}}{3 h_{y}^{\pm}}
    - \frac{h_y^{\pm}}{3 \alpha_{x}^{-} \alpha_{x}^{+} h_{x}}
\end{align}
The value of the center node coefficient is:
\begin{equation*}
    A_{(0,0)(0,0)} =
    \frac{4 k^{2} h_{x} h_{y} }{9}
    - \frac{4 h_{x}}{3 \alpha_{y}^{-} \alpha_{y}^{+} h_{y}}
    - \frac{4 h_{y}}{3 \alpha_{x}^{-} \alpha_{x}^{+} h_{x}}
\end{equation*}

\subsubsection{Numerical solution}
We assume that the numerical solution can be defined by $u^h(x) = C e^{i(k_x^h x + k_y^h y)}$
with two numerical wavenumbers $k_x^h = k^h cos{\theta}$ and $k_y^h = k^h sin{\theta}$.

\subsubsection{Dispersion relation}
The numerical wave number $k^h$ is asymptotically linked to the exact wavenumber when !!!:
\begin{equation}
    k^h \approx !!!
\end{equation}

\textit{Proof.}
Substituting the nodal numerical values in the dispertion relation \cref{eq:stenicl} yields a relation between that we can solve for $k^2$.
Due to this expression being long, we will use the following notations:
\begin{align*}
    &f_{\xi}^{\pm} = cos{\left(k_{\xi}^h h_{\xi}^{\pm}\right)} \text{ with } \xi \in \{x,y\} \\
    &g^{--} = cos{\left(h_x^- k_x^h + h_y^- k_y^h\right)} \\
    &g^{++} = cos{\left(h_x^+ k_x^h + h_y^+ k_y^h\right)} \\
    &g^{-+} = cos{\left(h_x^- k_x^h - h_y^+ k_y^h\right)} \\
    &g^{+-} = cos{\left(h_x^+ k_x^h - h_y^- k_y^h\right)}
\end{align*}
\tiny
\begin{equation}
    \label{eq:k2}
    k^2 = \frac{
        \kappa_0
        +
        \kappa_x^- f_{x}^{-} + \kappa_x^+ f_{x}^{+} + \kappa_y^- f_{y}^{-} + \kappa_y^+ f_{y}^{+}
        +
        \kappa^{--} g^{--} + \kappa^{++} g^{++} + \kappa^{-+} g^{-+} + \kappa^{+-} g^{+-}
    }
    {
        h_x^- h_x^+ h_y^- h_y^+ \left(
            \lambda_0
            +
            \lambda_x^- f_{x}^{-} + \lambda_x^+ f_{x}^{+} + \lambda_y^- f_{y}^{-} + \lambda_y^+ f_{y}^{+}
            +
            \lambda^{--} g^{--} + \lambda^{++} g^{++} + \lambda^{-+} g^{-+} + \lambda^{+-} g^{+-}
        \right)
    }
\end{equation}
\normalsize





\begin{equation*}
    k^2 =
    \frac{
        6 \left(
            2 f^{+}_{x} h^{-}_{x} \left(
                \left(h^{+}_{x}\right)^{2} h_{y}
                - \left(h^{+}_{y}\right)^{2} h^{-}_{y}
                - h^{+}_{y} \left(h^{-}_{y}\right)^{2}
            \right)
            + 2 f^{-}_{x} h^{+}_{x} \left(
                 \left(h^{-}_{x}\right)^{2} h_{y}
                - \left(h^{+}_{y}\right)^{2} h^{-}_{y}
                - h^{+}_{y} \left(h^{-}_{y}\right)^{2}
            \right)
            + 2 f^{+}_{y} h^{-}_{y} \left(
                h_{x} \left(h^{+}_{y}\right)^{2} 
                - \left(h^{+}_{x}\right)^{2} h^{-}_{x}
                - h^{+}_{x} \left(h^{-}_{x}\right)^{2}
            \right)
            + 2 f^{-}_{y} h^{+}_{y} \left(
                h_{x} \left(h^{-}_{y}\right)^{2}
                - h^{+}_{x} \left(h^{-}_{x}\right)^{2}
                - \left(h^{+}_{x}\right)^{2} h^{-}_{x}
            \right)
            - h^{+}_{x} h^{+}_{y} \left(h^{-}_{y}\right)^{2} g^{--}
            - h^{+}_{x} \left(h^{-}_{x}\right)^{2} h^{+}_{y} g^{--}
            - h^{+}_{x} \left(h^{-}_{x}\right)^{2} h^{-}_{y} g^{-+}
            - h^{+}_{x} \left(h^{+}_{y}\right)^{2} h^{-}_{y} g^{-+}
            - \left(h^{+}_{x}\right)^{2} h^{-}_{x} h^{+}_{y} g^{+-}
            - h^{-}_{x} h^{+}_{y} \left(h^{-}_{y}\right)^{2} g^{+-}
            - \left(h^{+}_{x}\right)^{2} h^{-}_{x} h^{-}_{y} g^{++}
            - h^{-}_{x} \left(h^{+}_{y}\right)^{2} h^{-}_{y} g^{++}
            + 2 \left(h^{+}_{x}\right)^{2} h^{-}_{x} h^{-}_{y}
            + 2 h^{+}_{x} h^{+}_{y} \left(h^{-}_{x}\right)^{2}
            + 2 h^{+}_{x} \left(h^{-}_{x}\right)^{2} h^{-}_{y}
            + 2 h^{+}_{x} \left(h^{+}_{y}\right)^{2} h^{-}_{y}
            + 2 h^{+}_{x} h^{+}_{y} \left(h^{-}_{y}\right)^{2}
            + 2 h^{-}_{x} \left(h^{+}_{y}\right)^{2} h^{-}_{y}
            \right)
        }
        {
            h^{+}_{x} h^{-}_{x} h^{+}_{y} h^{-}_{y}
            \left(
                  4 f^{+}_{x} h^{+}_{x} h_{y}
                + 4 f^{-}_{x} h^{-}_{x} h_{y}
                + 4 f^{+}_{y} h_{x} h^{+}_{y}
                + 4 f^{-}_{y} h_{x} h^{-}_{y}
                + 16 h_{x} h_{y}
                + h^{+}_{x} h^{+}_{y} g^{++}
                + h^{+}_{x} h^{-}_{y} g^{+-}
                + h^{-}_{x} h^{+}_{y} g^{-+}
                + h^{-}_{x} h^{-}_{y} g^{--}
            \right)
        }
\end{equation*}

\begin{equation}
    \gamma_0 + \sum_{\xi \in \{x, y\}} \sum_{i \in \{-, +\}} \gamma_{\xi}^i f_{\xi}^i + 
\end{equation}

Terme indépendant en haut
\begin{multline*}
    + 12 h^{-}_{x} h^{+}_{y} \left(\left(h^{+}_{x}\right)^{2} + \left(h^{-}_{y}\right)^{2}\right) \\
    + 12 h^{-}_{x} h^{-}_{y} \left( \left(h^{+}_{x}\right)^{2} + \left(h^{+}_{y}\right)^{2} \right) \\
    + 12 h^{+}_{x} h^{+}_{y} \left( \left(h^{-}_{x}\right)^{2} + \left(h^{-}_{y}\right)^{2} \right) \\
    + 12 h^{+}_{x} h^{-}_{y} \left( \left(h^{-}_{x}\right)^{2} + \left(h^{+}_{y}\right)^{2} \right)  \\
\end{multline*}
    
Termes en f en haut
\begin{multline*}
        12 f^{+}_{x} h^{-}_{x} \left(
            \left(h^{+}_{x}\right)^{2} h_{y}
            - \left(h^{+}_{y}\right)^{2} h^{-}_{y}
            - h^{+}_{y} \left(h^{-}_{y}\right)^{2}
        \right)
        + 12 f^{-}_{x} h^{+}_{x} \left(
                \left(h^{-}_{x}\right)^{2} h_{y}
            - \left(h^{+}_{y}\right)^{2} h^{-}_{y}
            - h^{+}_{y} \left(h^{-}_{y}\right)^{2}
        \right) \\
        + 12 f^{+}_{y} h^{-}_{y} \left(
            h_{x} \left(h^{+}_{y}\right)^{2} 
            - \left(h^{+}_{x}\right)^{2} h^{-}_{x}
            - h^{+}_{x} \left(h^{-}_{x}\right)^{2}
        \right)
        + 12 f^{-}_{y} h^{+}_{y} \left(
            h_{x} \left(h^{-}_{y}\right)^{2}
            - h^{+}_{x} \left(h^{-}_{x}\right)^{2}
            - \left(h^{+}_{x}\right)^{2} h^{-}_{x}
        \right)
\end{multline*}

Terme indépendant en bas (en facteur des 4h)
\begin{equation*}
    16 h_{x} h_{y}
\end{equation*}

Termes en f en bas (en facteur des 4h)
\begin{multline*}
    4 f^{+}_{x} h^{+}_{x} h_{y}
    + 4 f^{-}_{x} h^{-}_{x} h_{y}
    + 4 f^{+}_{y} h_{x} h^{+}_{y}
    + 4 f^{-}_{y} h_{x} h^{-}_{y}
\end{multline*}

Termes en g en bas (en facteur des 4h)
\begin{equation*}
                    + h^{+}_{x} h^{+}_{y} g^{++}
                + h^{+}_{x} h^{-}_{y} g^{+-}
                + h^{-}_{x} h^{+}_{y} g^{-+}
                + h^{-}_{x} h^{-}_{y} g^{--}
\end{equation*}

\end{document}