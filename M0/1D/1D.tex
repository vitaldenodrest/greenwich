\documentclass{article}

\usepackage{graphicx}
\usepackage{float}
\usepackage{amsmath}
\usepackage{amssymb}
\usepackage[
    colorlinks=true,
    linkcolor=blue,
    citecolor=red,
    urlcolor=cyan,
    filecolor=magenta,
]{hyperref}

\usepackage{cleveref}
\crefname{equation}{Eq.}{Eqs.}
\Crefname{equation}{Eq.}{Eqs.}
%\creflabelformat{equation}{#2#1#3}

\usepackage{fancyhdr}
\pagestyle{fancy}
\fancyhead[L]{V. de Nodrest, F. Magoulès, C. Lai, E. George}
\fancyhead[R]{}

\newtheorem{theorem}{Theorem}[subsection]

\begin{document}


\section{Problem statement}

\subsection{Strong formulation}
We will work on a one-dimensional Helmholtz problem.
Let $\Omega \subset \mathbb{R}$ be an open bounded region whose border $\delta \Omega$ is sufficiently smooth.
Our strong formulation is the following:
$$
\text{Find } u \text{ such that: }
\begin{cases}
\Delta u + k^2 u = -f & \text{in } \Omega \\
u = 0 & \text{on } \partial \Omega
\end{cases}
$$
$\Delta = \nabla^2$ is the Laplace operator, $u$ could be the unknown acoustic pressure field, $k$ is the wavenumber and $f$ is a source term.
In general, $u$ and $k$ can be complex-valued. However, the parameter studies will be conducted on real-valued cases, with $k>0$.
The extension to complex-valued cases should be straightforward.

\subsection{Galerkin formulation}
The usual Galerkin formulation for our problem is the following:
$$
\text{Find } u^h \in \mathcal{V}^h 
\text{ such that }
\forall v^h \in \mathcal{V}^h , ~
a_G(u^h, v^h) = \left\langle f, v^h  \right\rangle_{L^2(\Omega)}
$$
Where $\mathcal{V}^h \subset H_0^1(\Omega)$ is finite-dimensional,
$\left\langle u, v  \right\rangle_{L^2(\Omega)}$ is the $L^2(\Omega)$ scalar product,
and $a_G$ is the following sesquilinear form:
\begin{align*}
a_G: H_0^1(\Omega)^2 &\to \mathbb{C} \\
(u, v) &\mapsto \left\langle \nabla u, \nabla v  \right\rangle_{L^2(\Omega)} - k^2 \left\langle u, v \right\rangle_{L^2(\Omega)}
\end{align*}
We will use a mesh method to build $\mathcal{V}^h$, partitioning $\Omega$ in mutually exclusive elements.

\subsection{GLS formulation}

The Galerkin/least-squares formulation uses the previous formulation and adds other terms whose purpose is to minimize the square of the residue over the element interiors $\tilde{\Omega}$.
Their contribution is weighted by a parameter $\tau$ :
$$
\text{Find } u^h \in \mathcal{V}^h 
\text{, }
\forall v^h \in \mathcal{V}^h , ~
a_{GLS}(u^h, v^h) = \left\langle f, v^h \right\rangle_{L^2(\Omega)} + \left\langle \tau f, \mathcal{L}v^h \right\rangle_{L^2(\tilde{\Omega})}
$$
Where $\mathcal{L.} = \Delta. + k^2.$ is the Helmholtz operator, and $a_{GLS}$ is the follwing sesquilinear form:
\begin{align*}
a_{GLS}: H_0^1(\Omega)^2 &\to \mathbb{C} \\
(u, v) &\mapsto a_G(u,v) + \left\langle \tau \mathcal{L}u, \mathcal{L}v \right\rangle_{L^2(\tilde{\Omega})}
\end{align*}


\section{Dispersion analysis and optimal parameter}
The Helmholtz equation suffers from the pollution effect,
causing numerical wavenumbers to lose accuracy with the physical wavenumber increasing.

\paragraph{Exact solutions}
When considering the sourceless ($f=0$) and free-space version of our problem,
the exact solutions are $\forall x \in \mathbb{R}, ~ u(x) = C e^{ikx}$.
Our studies will revolve around this case.

\subsection{Dispersion analysis}

\paragraph{Numerical resolution}
\label{GalerkineNumRes}
Using linear interpolation with nodal shape functions $N_i$ that respect the partition of unity and Kronecker delta properties yields the linear system
$AU^h=0$,
where $U^h$ is a the vector of nodal values and $A$ is the impedence matrix. \\
When using our Galerkine formulation, the matrix is
$A_{G,ij} = \sum_{e \in E} K_{ij}^e - k^2 M_{ij}^e$
where $E$ is the set of elements whose boundary contain both nodes $i$ and $j$,
and $K_{ij}^e$ and $M_{ij}^e$
are respectively stiffness and masse terms assembled using shape functions $i$ and $j$ over the element $e$.

\paragraph{Dispersion relation}
\label{GalerkineDispRel}
We study, without loss of generality,
the stencil formed by an interior node $0$ and its neighbours of indexes $-1$ et $+1$.
The distances between those nodes are $h_x^-$ and $h_x^+$, as shown in the figure below:

\begin{figure}[h]
    \center
    \includegraphics[width=0.5\textwidth]{mesh_tikz.pdf}
    \caption{Node $0$ and its neighbors}
\end{figure}

For convenience, we might use
$h_{x}^{-} = \alpha_x^{-} h$, $h_{x}^{+} = \alpha_x^{+} h$ and $2h = h_{x}^- + h_{x}^+$. \\
The "dispersion relation" is the line $0$ of the system:
\begin{equation}
A_{0, -1}U_{-1}^h + A_{0, 0}U_{0}^h + A_{0, +1}U_{+1}^h = 0 \label{eq:stenicl}
\end{equation}
The values of the coefficients are:
\begin{align}
    A_{0, -1} &= \frac{h_{x}^{-} k^{2}}{6} + \frac{1}{h_{x}^{-}} \\
    A_{0, 0} &= \frac{2h k^{2}}{3} - \frac{2}{\alpha_{x}^{-} \alpha_{x}^{-} h} \\
    A_{0, +1} &= \frac{h_{x}^{+} k^{2}}{6} + \frac{1}{h_{x}^{+}}
\end{align}

\paragraph{Numerical solution}
We assume that the numerical solution can be defined by $u^h(x) = C e^{ik^{h}x}$ where $k^h$ is the numerical wavenumber.

% !!!
\begin{theorem}[Dispersion relation]
    The numerical wave number $k^h$ is asymptotically linked to the exact wavenumber $k$ when $k^h h \to 0$ and $k h \to 0$:
    \begin{equation}
        k^h \approx k - \frac{1}{24} \left({\alpha_x^{-}}^2 + {\alpha_x^{+}}^2 - \alpha_x^{-} \alpha_x^{+}\right) k^3 h^2
    \end{equation}
\end{theorem}

\textit{Proof.}
Substituting the nodal numerical values in the dispertion relation \cref{eq:stenicl} yields a relation between $k$ and $k^h$ that we can solve for $k^2$:
\begin{equation}
    \label{eq:k2}
    k^2 = \frac{6}{h_{x}^{+}h_{x}^{-}} ~
    \frac{\alpha_{x}^{-} \left(1 - \cos{h_{x}^{+} k^{h}}  \right) + \alpha_{x}^{+} \left(1 - \cos{h_{x}^{-} k^{h}}  \right)}
    {\alpha_{x}^{-} \left( 2 + \cos{h_{x}^{-} k^{h}} \right) + \alpha_{x}^{+} \left( 2 + \cos{h_{x}^{+} k^{h}} \right)}
\end{equation}
However, it cannot be exressed as $k^h = f(k)$ because of the cosines of varying wavenumbers.
We need to use a Taylor expansion on the cosines when $h k^h \to 0$:
\begin{equation}
    \cos{\left(\alpha_x^{\pm} h k^{h} \right)} = 1 - \frac{{\alpha_x^{\pm}}^2}{2} (h k^k)^2 + \frac{{\alpha_x^{\pm}}^4}{24} (h k^h)^4 + O((hk^h)^6)
\end{equation}
Substituting this expansion in \cref{eq:k2} and simplifying by common factors yields
an expression for $(k h)^2$ as a ratio of polynomials. It can be rewritten as:
\begin{align*}
    &(k h)^2 = X \frac{1}{1-Y} \\
    &X = (h k^{h})^2 + \chi (h k^{h})^4 + O((h k^{h})^6) \\
    &Y = \gamma_1 (h k^{h})^2 + \gamma_2 (h k^{h})^4 + O((h k^{h})^6) \\
    &\chi= - \frac{1}{12} \left({\alpha_x^{-}}^2 + {\alpha_x^{+}}^2 - \alpha_x^{-} \alpha_x^{+}\right) \\
    &\gamma_1 = \frac{1}{6} \left({\alpha_x^{-}}^2 + {\alpha_x^{+}}^2 - \alpha_x^{-} \alpha_x^{+}\right)  =-2 \chi \\
    &\gamma_2 = - \frac{1}{72} \frac{{\alpha_x^{-}}^5 + {\alpha_x^{+}}^5}{\alpha_x^{-} + \alpha_x^{+}}
\end{align*}
$Y \to 0$ as $h k^{h} \to 0$, thus allowing the following Taylor expansion:
\begin{equation}
    \label{eq:YY}
    \frac{1}{1-Y} = 1 + \gamma_1 (h k^{h})^2 + (\gamma_1^2 + \gamma_2) (h k^{h})^4 + O((h k^{h})^6)
\end{equation}
A polynomial expression of $(k h)^2$ is now available as a product of $X$ and $\frac{1}{1-Y}$:
\begin{equation}
    (k h)^2 = (h k^{h})^2 - \chi (h k^{h})^4 + O((h k^{h})^6) \label{eq:khx2}
\end{equation}
We need an expansion of $(h k^h)^2$ as powers of $(kh)^2$.
This is possible thanks to a series reversion (cite).
The first step is writing the Taylor expansion we are aiming for:
\begin{equation}
    (k^h h)^2 \approx \alpha_1 ((k h)^{2})^1 + \alpha_2 ((k h)^{2})^2 \label{eq:cible} % Precision ?
\end{equation}
Then, substituting \cref{eq:cible} into \cref{eq:khx2} allows us to determine the right coefficients for the expansion:
\begin{equation}
    (k^h h)^2 \approx (k h)^2 + \chi (k h)^4 \label{eq:square}
\end{equation}
The last step is extracting the Taylor series of \cref{eq:square}, with $k^h h > 0$:
\begin{equation}
    k^h h \approx k h + \frac{1}{2} \chi (k h)^3
\end{equation}
This concludes the proof. $\square$



\subsection{Optimal GLS parameter}
Our goal is to find a value for the parameter $\tau$ for each line of the linear system
$AU^h=0$.
This parameter is designed with the goal of obtaining $k=k^h$ with the GLS method,
thus enforcing wavenumber and nodal exactness of the numerical solution.

\paragraph{GLS numerical resolution}
\label{GLSNumRes}
The procedure is similar to the one described in \ref{GalerkineNumRes}.
However, this time, we have to account for GLS terms during the assembly procedure.
As we are using linear shape functions, the Laplacian part of the residue vanishes.
The $k^2$ part remains and the GLS impendence matrix is the following:
$$A_{GLS,ij} = \sum_{e \in E} K_{ij}^e - k^2 (1 - \tau k^2) M_{ij}^e$$


\begin{theorem}[Optimal GLS parameter]
    The optimal GLS parameter $\tau_0$ associated with the center node is:
    $$
    \tau_0 k^2
    =
    1
    -
    \frac{6}{k^2 h_{x}^{-} h_{x}^{+}}
    \frac
    {
        \alpha_x^{-} \left( 1 - \cos{h_{x}^{+} k} \right)
        +
        \alpha_x^{+} \left( 1 - \cos{h_{x}^{-} k} \right)
    }
    {
        \alpha_x^{-} \left( 2 + \cos{h_{x}^{-} k} \right)
        +
        \alpha_x^{+} \left( 2 + \cos{h_{x}^{+} k} \right)
    }
    $$
\end{theorem}

\textit{Proof.}
The most straightforward approach to computing this parameter is writing the GLS dispersion relation as in \ref{GalerkineDispRel} by
replacing $k^2$ with $k^2 (1 - \tau k^2)$, as demonstrated in \ref{GLSNumRes}.
Then, setting all wavenumbers to $k$, enforces numerical solution exactness.
Finally, solving for $\tau$ yields the parameter making this possible. \\
One might notice that it is possible to find a shortcut using \cref{eq:k2}.

\end{document}