\documentclass{article}

\usepackage{amsmath}
\usepackage{amssymb}

\begin{document}

\paragraph{Exact solution.}
We consider the sourceless Helmholtz equation in homogeneous media
$
\Delta u + k^2 u = 0
$
in 
$
\mathbb{R}
$.
The solutions are defined by
$
u(x) = c e^{ikx}
$
with any constant $c$.

\paragraph{Numerical solution}
%Using linear interpolation yields a linear system $A U^h = 0$ where

We discretize the 1D domain and consider, without loss of generality, the stencil formed by an interior node $0$ and its neighbours $-1$ et $+1$.
The "dispersion relation" linear equation between $U^h_0$ and the neighboring nodal values is:
\begin{equation}
A_{0, -1}U_{-1} + A_{0, 0}U_{0} + A_{0, +1}U_{+1} = 0 \label{eq:stenicl}
\end{equation}



\begin{align*}
    A_{0, -1} &= \frac{h_{x}^{-} k^{2}}{6} + \frac{1}{h_{x}^{-}} \\
    A_{0, 0} &= \frac{h_{x}^{-} k^{2}}{3} + \frac{h_{x}^{+} k^{2}}{3} - \frac{1}{h_{x}^{-}} - \frac{1}{h_{x}^{+}} \\
    A_{0, 1} &= \frac{h_{x}^{+} k^{2}}{6} + \frac{1}{h_{x}^{+}}
\end{align*}

\section{Dispersion analysis}
We assume that the numerical solution can be defined by $u^h(x) = c e^{ik^{h}x}$ where $k^h$ is the numerical wavenumber.
$h_{x}^{-} = \alpha^{-} h_x$ and $h_{x}^{+} = \alpha^{+} h_x$

\paragraph{Theorem 1.1}
For the 1D Gakerine method stencil with linear finite elements of varying sizes, the numerical wave number $k^h$ is linked to the exact wavenumber when $k^h h \to 0$ and $k h \to 0$:
\begin{equation}
    k^h \approx k - \frac{1}{24} \left({\alpha^{-}}^2 + {\alpha^{+}}^2 - \alpha^{-} \alpha^{+}\right) k^3 h_x^2
\end{equation}

\textit{Proof.}
Substituting the nodel numerical values in the dispertion relation \eqref{eq:stenicl} yields a relation between $k$ and $k^h$ that we can solve for $k$:
\begin{equation}
    \label{eq:k2}
    k^2 = \frac{6}{h_{x}^{+}h_{x}^{-}} ~
    \frac{h_{x}^{-} \left(1 - \cos{\left(h_{x}^{+} k^{h} \right)}  \right) + h_{x}^{+} \left(1 - \cos{\left(h_{x}^{-} k^{h} \right)}  \right)}
    {h_{x}^{-} \left( 2 + \cos{\left(h_{x}^{-} k^{h} \right)} \right) + h_{x}^{+} \left( 2 + \cos{\left(h_{x}^{+} k^{h} \right)} \right)}
\end{equation}
However, it cannot be written as $k^h = f(k)$ because of the cosines of varying wavenumbers.
We need to use a Taylor expansion on the cosines when $h_x k^h \to 0$:
\begin{equation}
    \cos{\left(\alpha^{\pm} h_{x} k^{h} \right)} = 1 - \frac{{\alpha^{\pm}}^2}{2} (h_x k^k)^2 + \frac{{\alpha^{\pm}}^4}{24} (h_x k^h)^4 + O((h_x k^h)^6)
\end{equation}
Substituting this expansion in \eqref{eq:k2} and simplifying by common factors yields:
\small
\begin{equation*}
    k^2 h_x^2 =
    \frac{
        6 \left(
        12\left(\alpha^{-} + \alpha^{+}\right) (h_x k^{h})^2
        -
        \left({\alpha^{-}}^3 + {\alpha^{+}}^3\right) (h_x k^{h})^4
        \right)
        + O((h_x k^{h})^6) % A vérifier
    }
    {
        72 \left(\alpha^{-} + \alpha^{+}\right)
        -
        12\left({\alpha^{-}}^3 + {\alpha^{+}}^3 \right) (h_x k^{h})^2
        + 
        \left({\alpha^{-}}^5 + {\alpha^{+}}^5 \right) (h_x k^{h})^4
        + + O((h_x k^{h})^6) % A vérifier
    }
\end{equation*} % !!!
\normalsize
This ratio of polynomials is inconvenient. Let's rewrite $(k h_x)^2$:
\begin{align}
    &(k h_x)^2 = X \frac{1}{1-Y} \\
    &X = (h_x k^{h})^2 + \chi (h_x k^{h})^4 + O((h_x k^{h})^6) \\ \label{eq:X}
    &Y = \gamma_1 (h_x k^{h})^2 + \gamma_2 (h_x k^{h})^4 + O((h_x k^{h})^6) \\
    &\chi= - \frac{1}{12} \left({\alpha^{-}}^2 + {\alpha^{+}}^2 - \alpha^{-} \alpha^{+}\right) \\
    &\gamma_1 = \frac{1}{6} \left({\alpha^{-}}^2 + {\alpha^{+}}^2 - \alpha^{-} \alpha^{+}\right)  =-2 \chi \\
    &\gamma_2 = - \frac{1}{72} \frac{{\alpha^{-}}^5 + {\alpha^{+}}^5}{\alpha^{-} + \alpha^{+}}
\end{align}
$Y \to 0$ as $h_x k^{h} \to 0$, thus yielding the following Taylor expansion:
\begin{equation}
    \label{eq:YY}
    \frac{1}{1-Y} = 1 + \gamma_1 (h_x k^{h})^2 + (\gamma_1^2 + \gamma_2) (h_x k^{h})^4 + O((h_x k^{h})^6)
\end{equation}
A polynomial expression of $(k h_x)^2$ is now available as a product of \eqref{eq:X} and \eqref{eq:YY}:
\begin{equation}
    (k h_x)^2 = (h_x k^{h})^2 - \chi (h_x k^{h})^4 + O((h_x k^{h})^6) \label{eq:khx2}
\end{equation}
We need an expansion of $(h_xk^h)^2$ as powers of $(kh)^2$. This is possible thanks to a series reversion (cite). The first step is writing the Taylor expansion we are trying to achieve:
\begin{equation}
    (k^h h_x)^2 \approx \alpha_1 ((k h_x)^{2})^1 + \alpha_2 ((k h_x)^{2})^2 \label{eq:cible} % Precision ?
\end{equation}
Then, substituting \eqref{eq:cible} into \eqref{eq:khx2} allows us to determine the right coefficients for the expansion:
\begin{equation}
    (k^h h_x)^2 \approx (k h_x)^2 + \chi (k h_x)^4 \label{eq:square}
\end{equation}
The last step is extracting the Taylor series of \eqref{eq:square}, which is made under the hypothesis that $k^h h_x > 0$:
\begin{equation}
    k^h h_x \approx k h_x + \frac{1}{2} \chi (k h_x)^3
\end{equation}
This concludes the proof. $\square$


\section{Optimal GLS parameter}

$$
\tau k^2
=
1
-
\frac{6}{k^2 h_{x}^{-} h_{x}^{+}}
\frac
{
    \alpha^{-} \left( 1 - \cos{\left(h_{x}^{+} k \right)} \right)
    +
    \alpha^{+} \left( 1 - \cos{\left(h_{x}^{-} k \right)} \right)
}
{
    \alpha^{-} \left( 2 + \cos{\left(h_{x}^{-} k \right)} \right)
    +
    \alpha^{+} \left( 2 + \cos{\left(h_{x}^{+} k \right)} \right)
}
$$




\end{document}