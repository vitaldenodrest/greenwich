\documentclass{article}

\usepackage{amsmath}
\usepackage{amssymb}

\begin{document}

\paragraph{Strong formulation}
We will work on a one-dimensional Helmholtz problem.
Let $\Omega \subset \mathbb{R}$ be an open bounded region whose border $\delta \Omega$ is sufficiently smooth.
Our strong formulation is the following:
$$
\text{Find } u \text{ such that: }
\begin{cases}
\Delta u + k^2 u = -f & \text{in } \Omega \\
u = 0 & \text{on } \partial \Omega
\end{cases}
$$
$\Delta = \nabla^2$ is the Laplace operator, $u$ could be the unknown acoustic pressure field, $k$ is the wavenumber and $f$ is a source term.
In general, $u$ and $k$ can be complex-valued. However, this study will be conducted on real-valued cases, with $k>0$.

\paragraph{Galerkin formulation}
The usual Galerkin formulation for our problem is the following:
$$
\text{Find } u^h \in \mathcal{V}^h 
\text{ such that }
\forall v^h \in \mathcal{V}^h , ~
a_G(u^h, v^h) = \left\langle f, v^h  \right\rangle_{L^2(\Omega)}
$$
Where $\mathcal{V}^h \subset H_0^1(\Omega)$ is finite-dimensional,
$\left\langle u, v  \right\rangle_{L^2(\Omega)}$ is the $L^2(\Omega)$ scalar product,
and $a_G$ is the following sesquilinear form:
\begin{align*}
a_G: H_0^1(\Omega)^2 &\to \mathbb{C} \\
(u, v) &\mapsto \left\langle \nabla u, \nabla v  \right\rangle_{L^2(\Omega)} - k^2 \left\langle u, v \right\rangle_{L^2(\Omega)}
\end{align*}
We will use a mesh method to build $\mathcal{V}^h$, partitioning $\Omega$ in mutually exclusive elements.

\paragraph{GLS}
The Galerkin/least-squares formulation uses the previous formulation and adds other terms whose purpose is to minimize the square of the residue over the element interiors $\tilde{\Omega}$.
Their contribution is weighted by a parameter $\tau$ :
$$
\text{Find } u^h \in \mathcal{V}^h 
\text{, }
\forall v^h \in \mathcal{V}^h , ~
a_{GLS}(u^h, v^h) = \left\langle f, v^h \right\rangle_{L^2(\Omega)} + \left\langle \tau f, \mathcal{L}v^h \right\rangle_{L^2(\tilde{\Omega})}
$$
Where $\mathcal{L.} = \Delta. + k^2.$ is the Helmholtz operator, and $a_{GLS}$ is the follwing sesquilinear form:
\begin{align*}
a_{GLS}: H_0^1(\Omega)^2 &\to \mathbb{C} \\
(u, v) &\mapsto a_G(u,v) + \left\langle \tau \mathcal{L}u, \mathcal{L}v \right\rangle_{L^2(\tilde{\Omega})}
\end{align*}

\section{Dispersion analysis}
The Helmholtz equation suffers from the pollution effect,
causing numerical wavenumbers to lose accuracy with the physical wavenumber increasing.
In this section, 

\paragraph{Exact solutions}
When considering the sourceless ($f=0$) and free-space version of our problem,
the exact solutions are:
$
\forall x \in \mathbb{R}, ~ u(x) = C e^{ikx}
$

\paragraph{Numerical resolution}
Using linear interpolation with nodal shape functions $N_i$ that respect the partition of unity and Kronecker delta properties yields the linear system
$AU^h=0$,
where $U^h$ is a the vector of nodal values and $A$ is the impedence matrix. \\
When using our Galerkine formulation, the matrix is
$A_{G,ij} = \sum_{e \in E} K_{ij}^e - k^2 M_{ij}^e$
where $E$ is the set of elements whose boundary contain both nodes $i$ and $j$,
and $K_{ij}^e$ and $M_{ij^e}$
are respectively stiffness and masse terms assembled using shape functions $i$ and $j$ over the element $e$.

\paragraph{Dispersion relation}
We consider, without loss of generality, the stencil formed by an interior node $0$ and its neighbours $-1$ et $+1$.
The "dispersion relation" linear equation between $U^h_0$ and the neighboring nodal values is:
\begin{equation}
A_{0, -1}U_{-1}^h + A_{0, 0}U_{0}^h + A_{0, +1}U_{+1}^h = 0 \label{eq:stenicl}
\end{equation}



\begin{align*}
    A_{0, -1} &= \frac{h_{x}^{-} k^{2}}{6} + \frac{1}{h_{x}^{-}} \\
    A_{0, 0} &= \frac{h_{x}^{-} k^{2}}{3} + \frac{h_{x}^{+} k^{2}}{3} - \frac{1}{h_{x}^{-}} - \frac{1}{h_{x}^{+}} \\
    A_{0, 1} &= \frac{h_{x}^{+} k^{2}}{6} + \frac{1}{h_{x}^{+}}
\end{align*}


We assume that the numerical solution can be defined by $u^h(x) = c e^{ik^{h}x}$ where $k^h$ is the numerical wavenumber.
$h_{x}^{-} = \alpha^{-} h_x$ and $h_{x}^{+} = \alpha^{+} h_x$

\paragraph{Theorem 1.1}
For the 1D Gakerine method stencil with linear finite elements of varying sizes, the numerical wave number $k^h$ is linked to the exact wavenumber when $k^h h \to 0$ and $k h \to 0$:
\begin{equation}
    k^h \approx k - \frac{1}{24} \left({\alpha^{-}}^2 + {\alpha^{+}}^2 - \alpha^{-} \alpha^{+}\right) k^3 h_x^2
\end{equation}

\textit{Proof.}
Substituting the nodel numerical values in the dispertion relation \eqref{eq:stenicl} yields a relation between $k$ and $k^h$ that we can solve for $k$:
\begin{equation}
    \label{eq:k2}
    k^2 = \frac{6}{h_{x}^{+}h_{x}^{-}} ~
    \frac{h_{x}^{-} \left(1 - \cos{\left(h_{x}^{+} k^{h} \right)}  \right) + h_{x}^{+} \left(1 - \cos{\left(h_{x}^{-} k^{h} \right)}  \right)}
    {h_{x}^{-} \left( 2 + \cos{\left(h_{x}^{-} k^{h} \right)} \right) + h_{x}^{+} \left( 2 + \cos{\left(h_{x}^{+} k^{h} \right)} \right)}
\end{equation}
However, it cannot be written as $k^h = f(k)$ because of the cosines of varying wavenumbers.
We need to use a Taylor expansion on the cosines when $h_x k^h \to 0$:
\begin{equation}
    \cos{\left(\alpha^{\pm} h_{x} k^{h} \right)} = 1 - \frac{{\alpha^{\pm}}^2}{2} (h_x k^k)^2 + \frac{{\alpha^{\pm}}^4}{24} (h_x k^h)^4 + O((h_x k^h)^6)
\end{equation}
Substituting this expansion in \eqref{eq:k2} and simplifying by common factors yields:
\small
\begin{equation*}
    k^2 h_x^2 =
    \frac{
        6 \left(
        12\left(\alpha^{-} + \alpha^{+}\right) (h_x k^{h})^2
        -
        \left({\alpha^{-}}^3 + {\alpha^{+}}^3\right) (h_x k^{h})^4
        \right)
        + O((h_x k^{h})^6) % A vérifier
    }
    {
        72 \left(\alpha^{-} + \alpha^{+}\right)
        -
        12\left({\alpha^{-}}^3 + {\alpha^{+}}^3 \right) (h_x k^{h})^2
        + 
        \left({\alpha^{-}}^5 + {\alpha^{+}}^5 \right) (h_x k^{h})^4
        + + O((h_x k^{h})^6) % A vérifier
    }
\end{equation*} % !!!
\normalsize
This ratio of polynomials is inconvenient. Let's rewrite $(k h_x)^2$:
\begin{align}
    &(k h_x)^2 = X \frac{1}{1-Y} \\
    &X = (h_x k^{h})^2 + \chi (h_x k^{h})^4 + O((h_x k^{h})^6) \\ \label{eq:X}
    &Y = \gamma_1 (h_x k^{h})^2 + \gamma_2 (h_x k^{h})^4 + O((h_x k^{h})^6) \\
    &\chi= - \frac{1}{12} \left({\alpha^{-}}^2 + {\alpha^{+}}^2 - \alpha^{-} \alpha^{+}\right) \\
    &\gamma_1 = \frac{1}{6} \left({\alpha^{-}}^2 + {\alpha^{+}}^2 - \alpha^{-} \alpha^{+}\right)  =-2 \chi \\
    &\gamma_2 = - \frac{1}{72} \frac{{\alpha^{-}}^5 + {\alpha^{+}}^5}{\alpha^{-} + \alpha^{+}}
\end{align}
$Y \to 0$ as $h_x k^{h} \to 0$, thus yielding the following Taylor expansion:
\begin{equation}
    \label{eq:YY}
    \frac{1}{1-Y} = 1 + \gamma_1 (h_x k^{h})^2 + (\gamma_1^2 + \gamma_2) (h_x k^{h})^4 + O((h_x k^{h})^6)
\end{equation}
A polynomial expression of $(k h_x)^2$ is now available as a product of \eqref{eq:X} and \eqref{eq:YY}:
\begin{equation}
    (k h_x)^2 = (h_x k^{h})^2 - \chi (h_x k^{h})^4 + O((h_x k^{h})^6) \label{eq:khx2}
\end{equation}
We need an expansion of $(h_xk^h)^2$ as powers of $(kh)^2$. This is possible thanks to a series reversion (cite). The first step is writing the Taylor expansion we are trying to achieve:
\begin{equation}
    (k^h h_x)^2 \approx \alpha_1 ((k h_x)^{2})^1 + \alpha_2 ((k h_x)^{2})^2 \label{eq:cible} % Precision ?
\end{equation}
Then, substituting \eqref{eq:cible} into \eqref{eq:khx2} allows us to determine the right coefficients for the expansion:
\begin{equation}
    (k^h h_x)^2 \approx (k h_x)^2 + \chi (k h_x)^4 \label{eq:square}
\end{equation}
The last step is extracting the Taylor series of \eqref{eq:square}, which is made under the hypothesis that $k^h h_x > 0$:
\begin{equation}
    k^h h_x \approx k h_x + \frac{1}{2} \chi (k h_x)^3
\end{equation}
This concludes the proof. $\square$


\section{Optimal GLS parameter}

$$
\tau k^2
=
1
-
\frac{6}{k^2 h_{x}^{-} h_{x}^{+}}
\frac
{
    \alpha^{-} \left( 1 - \cos{\left(h_{x}^{+} k \right)} \right)
    +
    \alpha^{+} \left( 1 - \cos{\left(h_{x}^{-} k \right)} \right)
}
{
    \alpha^{-} \left( 2 + \cos{\left(h_{x}^{-} k \right)} \right)
    +
    \alpha^{+} \left( 2 + \cos{\left(h_{x}^{+} k \right)} \right)
}
$$




\end{document}